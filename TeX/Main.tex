\documentclass[titlepage, landscape, a4paper, 12pt, twocolumn]{jsarticle}

\usepackage[top=15truemm,bottom=15truemm,left=10truemm,right=10truemm]{geometry}

\usepackage{ascmac,here,txfonts}
\usepackage{listings}
\usepackage{color}

\usepackage{bm}
\usepackage{mathrsfs}
\usepackage{amssymb}

\definecolor{dkgreen}{rgb}{0,0.5,0}
\definecolor{purple}{rgb}{0.5,0,0.5}
\definecolor{comcol}{rgb}{0.8,0.4,0}

\lstset{
  language={C++},
  basicstyle={\footnotesize\ttfamily},
  identifierstyle={\footnotesize},
  commentstyle={\footnotesize\itshape\color{red}},
  keywordstyle={\footnotesize\bfseries\color{blue}},
  ndkeywordstyle={\footnotesize\bfseries\color{comcol}},
  morekeywords={void, bool, char, int, short, long, double, auto},
  morekeywords={vector, string, priority\_queue, pair, set, map},
  morendkeywords={if, else, while, for, do, break, continue, return},
  morendkeywords={operator, const, struct, class, private, public, typedef},
  morendkeywords={int_type, INF},
  morendkeywords={using, namespace},
  morendkeywords={rep, all},
  stringstyle={\footnotesize\color{purple}},
  frame={tb},
  breaklines=true,
  columns=[l]{fullflexible},
  numbers=left,
  xrightmargin=0ex,
  xleftmargin=3ex,
  numberstyle={\scriptsize},
  stepnumber=1,
  numbersep=1ex,
  lineskip=-0.5ex
}

\begin{document}
\section{Template}
\lstinputlisting{../Template.cpp}

\section{Graph}
\subsection{二部マッチング}
\lstinputlisting{../Graph/BipartiteMatching.cpp}
\subsection{ダイクストラ}
\lstinputlisting{../Graph/Dijkstra.cpp}
\subsection{最大流}
\lstinputlisting{../Graph/FordFulkerson.cpp}
\subsection{最小費用流}
\lstinputlisting{../Graph/MinimumCostFlow.cpp}
\subsection{最小共通祖先}
\lstinputlisting{../Graph/LCA.cpp}

\section{Math}
\subsection{繰り返し2乗法}
\lstinputlisting{../Math/ModPow.cpp}
\subsection{コンビネーション}
\lstinputlisting{../Math/Combination.cpp}

\section{DataStructure}
\subsection{Segment Tree}
\lstinputlisting{../DataStructure/SegTree.cpp}
\subsection{StarrySky Tree}
\lstinputlisting{../DataStructure/StarrySkyTree.cpp}
\subsection{Treap}
\lstinputlisting{../DataStructure/Treap.cpp}
\subsection{UnionFind}
\lstinputlisting{../DataStructure/UnionFind.cpp}


\section{String}
\subsection{ローリングハッシュ}
\lstinputlisting{../String/RollingHash.cpp}


\section{Geometry}
\subsection{幾何}
\lstinputlisting{../Geometry/Geometry.cpp}



%
%
\end{document}